LAD-4 launched from the Friends of Amateur Rocketry (FAR) site in the Mojave Desert near Cantil, CA on the afternoon of Saturday, March 7, 2020.The prevailing weather condition at the time of launch was strong winds, with wind speeds at 20 miles per hour to the northeast and gusts of 24 miles per hour; the recorded temperature was 67.5 degrees Fahrenheit (19.7 degrees Celsius). 
\newline\newline
The command to launch was given around 13:40 local time under partly cloudy skies. Powered ascent was nominal; the motor burned for 5.40 seconds (cf. nominal burn of 5.70 seconds). As observed on the ground and from in-flight footage, the rocket began to undergo pitch-roll locking (“coning”) shortly after launch at a frequency of approximately 0.25 Hz.  It is hypothesized that this is a result of slight asymmetry in the nose cone and small misalignments of the fins. At the end of burn, the rocket attained a velocity of 411.6 m/s (Mach 1.2), and reached an altitude of 2997 feet (913 m) above ground level. Strong winds at the time of launch caused the rocket to pitch due North into the wind slightly during ascent, reducing the apogee by approximately 15\% from the nominal altitude predicted by the trajectory simulation. The rocket’s flight was otherwise nominal during the powered ascent phase. Following motor burnout, the rocket continued into unpowered ascent and, 26.00 seconds after launch, reached its apogee of 11,193 feet (3411.6 m) above ground level. 
\newline\newline
At 26.05 seconds after launch, a signal to detonate the 10g drogue parachute charge was given by the Stratologger module. The charge was set off and the nylon bolts sheared as designed. The drogue parachute was ejected from the airframe and inflated. During the ejection event, the main parachute was inadvertently partially ejected from the airframe along with the drogue. The main parachute thankfully did not inflate due to the Tender Descender carabiners constraining it. During this process, the main parachute also began to tangle with itself due to the violence of the ejection event and surrounding atmospheric conditions.
\newline\newline
The rocket continued a nominal drogue descent for another 117.10 seconds until reaching an altitude of 700 feet above ground level (AGL). During this period, the partially ejected main parachute’s cords tangled further due to spinning of the airframe during descent. At the time the rocket reached 700 feet AGL, recorded at 143.15 seconds after launch, the Stratologger module issued a signal to fire the smaller (3g) charge inside the Tender Descender. The charge fired, and the Tender Descender performed nominally, releasing the carabiners that retained the top and bottom attachment points of the main parachute. Due to the tangling of the parachute cords, however, the main parachute only partially inflated at this stage, and descent continued at an essentially unchanged rate under the drogue parachute, substantially exceeding design limitations on landing velocity. The rocket touched down in the desert brush some 8.10 seconds later at a point 1.09 miles (1746.2 meters) east of the launch site, with a total flight time of 151.25 seconds. 
\newline\newline
The rocket airframe was located and recovered approximately one hour after landing. The airframe was intact, the nosecone and parachute system having landed a few feet away attached via the length of paracord. Due to the impact with terrain at the higher-than-expected velocity under the drogue parachute, the aft-most section of the airframe (approximately 4-6 inches of the slotted section) sustained moderate damage (cracks in the composite matrix) caused by shearing against the aluminum fins. The rest of the rocket was recovered in a stable, flyable condition. The parameters of flight recorded by the onboard computers were later corroborated using footage from in-flight cameras aboard the rocket.