LAD-4, ``\textit{Updog}'', is Space Enterprise at Berkeley's third fully-completed flight vehicle in the Low Altitude Demonstrator (LAD) program. As the fourth LAD airframe to be constructed overall, considerations regarding cost savings, time savings, and material efficiency were paramount. As a test vehicle, one critical aspect of LAD-4's airframe design was reusability and modularity; the rocket was designed to be flown up to ten times as necessary.  This led to the natural progression of LAD-4's design to allow for the ability to rapidly replace damaged parts, to house different motor configurations in future flights, and to integrate active aerodynamic control modules for further test projects. The rocket had a nominal, minimal outside diameter of 6.5 inches, a height of 90 inches, and was powered by an Aerotech M1340W-PS solid motor for its maiden flight. On March 7, 2020, at the Friends of Amateur Rocketry site in Cantil, California, LAD-4 flew to Mach 1.2 and an apogee of 11,193 feet above ground level, successfully establishing new UC Berkeley records for altitude and speed.  